\subsection{Command Line Options}

Usage: scummvm [OPTIONS]... [GAME]\\
\begin{tabular}{ll}
  [GAME]                  &Short name of game to load. For example, 'monkey'\\
                          &for Monkey Island. This can be either a built-in\\
                          &gameid, or a user configured target.\\
\\
  -v, --version           &Display ScummVM version information and exit\\
  -h, --help              &Display a brief help text and exit\\
  -z, --list-games        &Display list of supported games and exit\\
  -t, --list-targets      &Display list of configured targets and exit\\
\\
  -c, --config=CONFIG     &Use alternate configuration file\\
  -p, --path=PATH         &Path to where the game is installed\\
  -x, --save-slot[=NUM]   &Save game slot to load (default: autosave)\\
  -f, --fullscreen        &Force full-screen mode\\
  -F, --no-fullscreen     &Force windowed mode\\
  -g, --gfx-mode=MODE     &Select graphics scaler (see also section 5.3)\\
  -e, --music-driver=MODE &Select music driver (see also section 7.0)\\
  -q, --language=LANG     &Select language (see also section 5.2)\\
  -m, --music-volume=NUM  &Set the music volume, 0-255 (default: 192)\\
  -o, --master-volume=NUM &Set the master volume, 0-255 (default: 192)\\
  -s, --sfx-volume=NUM    &Set the sfx volume, 0-255 (default: 192)\\
  -r, --speech-volume=NUM  &Set the voice volume, 0-255 (default: 192)\\
  -n, --subtitles         &Enable subtitles (use with games that have voice)\\
  -b, --boot-param=NUM    &Pass number to the boot script (boot param)\\
  -d, --debuglevel=NUM    &Set debug verbosity level\\
  -u, --dump-scripts      &Enable script dumping if a directory called\\
                          &'dumps' exists in the current directory\\
\\
  --cdrom=NUM             &CD drive to play CD audio from\\
                          &(default: 0 = first drive)\\
  --joystick[=NUM]        &Enable input with joystick (default: 0 = first\\
                          &joystick)\\
  --platform=WORD         &Specify version of game (allowed values: amiga,\\
                          &atari, mac, pc)\\
  --multi-midi            &Enable combination of Adlib and native MIDI\\
  --native-mt32           &True Roland MT-32 (disable GM emulation)\\
  --aspect-ratio          &Enable aspect ratio correction\\
\\
  --alt-intro             &Use alternative intro for CD versions of Beneath a\\
                          &Steel Sky and Flight of the Amazon Queen\\
  --copy-protection       &Enable copy protection in SCUMM games, when\\
                          &ScummVM disables it by default.\\
  --demo-mode             &Start demo mode of Maniac Mansion (Classic version)\\
  --tempo=NUM             &Set music tempo (in percent, 50-200) for SCUMM\\
                          &games (default: 100)\\
  --talkspeed=NUM         &Set talk speed for SCUMM games (default: 60)\\
\end{tabular}

The meaning of most long options can be inverted by prefixing them with "no-",
e.g. --no-aspect-ratio. This is useful if you want to override a setting in the
configuration file.

The short game name ('game target') you see at the end of the command
line is very important. A short list is contained at the top of this
file. You can also get the current list of games and game names at:

\begin{center}
  \url{http://www.scummvm.org/compatibility.php}
\end{center}

Examples:
\begin{itemize}
\item Win32:\\
Running Monkey Island, fullscreen, from a hard disk:
\begin{verbatim}
C:\Games\scummvm.exe -f -pC:\Games\monkey\ monkey
\end{verbatim}
  Running Full Throttle from CD, fullscreen and with subtitles enabled:
\begin{verbatim}
C:\Games\scummvm.exe -f -n -pD:\resource\ ft
\end{verbatim}
 \item Unix:\\
  Running Monkey Island, fullscreen, from a hard disk:
\begin{verbatim}
/path/to/scummvm -f -p/games/LucasArts/monkey/ monkey
\end{verbatim}
  Running Full Throttle from CD, fullscreen and with subtitles enabled:
\begin{verbatim}
/path/to/scummvm -f -n -p/cdrom/resource/ ft
\end{verbatim}
\end{itemize}
%%% Local Variables: 
%%% mode: latex
%%% TeX-master: "readme"
%%% End: 
