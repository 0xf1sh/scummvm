%%% Local Variables: 
%%% mode: latex
%%% TeX-master: "readme"
%%% End: 

\subsection{Output sample rate}

The output sample rate tells ScummVM how many sound samples to play per channel
per second. There is much that could be said on this subject, but most of it
would be irrelevant here. The short version is that for most games 22050 Hz is
fine, but in some cases 44100 Hz is preferable. On extremely low-end systems
you may want to use 11025 Hz, but it's unlikely that you have to worry about
that.

To elaborate, most of the sounds ScummVM has to play were sampled at either
22050 Hz or 11025 Hz. Using a higher sample rate will not magically improve the
quality of these sounds. Hence, 22050 Hz is fine.

Some games use CD audio. If you use compressed files for this, they are
probably sampled at 44100 Hz, so for these games that may be a better choice of
sample rate.

When using the Adlib, FM Towns, PC Speaker or IBM PCjr music drivers, ScummVM
is responsible for generating the samples. Usually 22050 Hz will be plenty for
these, but there is at least one piece of Adlib music in Beneath a Steel Sky
that will sound a lot better at 44100 Hz.

Using frequencies in between is not recommended. For one thing, your sound card
may not support it. In theory, ScummVM should fall back on a sensible frequency
in that case, but don't count on it. More importantly, ScummVM has to resample
all sounds to its output frequency. This is much easier to do well if the
output frequency is a multiple of the original frequency.
