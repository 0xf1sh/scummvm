
%%% Local Variables: 
%%% mode: latex
%%% TeX-master: "readme"
%%% End: 

\section{Compiling}

You need SDL-1.2.2 or newer (older versions may work, but are unsupported), and
a supported compiler. Several compilers, including GCC, mingw and Microsoft
Visual C++ are supported. If you wish to use MP3-compressed CD tracks or
.SOU files, you will need to install the MAD library and define
USE\_MAD. Tools for compressing .SOU files to .SO3 files can be
found in the 'tools' CVS module, or in the 'scummvm-tools' package.\\
~\\
You can also comment/uncomment appropriate lines in the build.rules file to
use sdl\_gl.cpp instead of sdl.cpp. This allows hardware accelerated bilinear
filtering by using OpenGL textures.\\
~\\
On Win9x/NT/XP you can define USE\_WINDBG and attach WinDbg to browse debug 
messages (see http://www.sysinternals.com/ntw2k/freeware/debugview.shtml).\\
GCC:
  \begin{itemize}
  \item Type ./configure
  \item Type make (or gmake if that's what GNU make is called on your
        system) and hopefully ScummVM will compile for you.
  \end{itemize}
MingW -  Windows 5/98/ME/NT/2000/XP/2003:
  \begin{itemize}
  \item Open Makefile.mingw, alter SDL paths and choose compiling
    options.
  \item Type make -f Makefile.mingw,  hopefully ScummVM will compile for you.
  \end{itemize}
MS Visual C++:
  \begin{itemize}
  \item Open the workspace, scummwm.dsw
  \item Enter the path to the SDL include files in
    Tools|Options|Directories
  \item Now it should compile successfully. 
  \end{itemize}
PocketPC Windows CE:
  \begin{itemize}
  \item Download the SDLAudio library:\\
    http://arisme.free.fr/PocketScumm/sources/SDLAudio-1.2.3-src.zip
  \item Open and compile the SDLAudio WCEBuild/WCEBuild workspace in
    EVC++
  \item Open the ScummVM wince/PocketScumm workspace
  \item Enter the SDLAudio directory to your includes path
  \item Enter the compiled SDLAudio.lib to your link libraries list
  \item Now it should compile successfully
  \end{itemize}
Debian GNU/Linux:
  \begin{itemize}
  \item Install the packages 'build-essential', 'fakeroot', 'debhelper',
          and 'libsdl1.2-dev' on your system.
  \item nstall any of these packages (optional): 'libvorbis-dev' (for Ogg
          Vorbis support), 'libasound2-dev' (for ALSA sequencer support),
          'libmad0-dev' (for MAD MP3 support), 'zlib1g-dev' (for compressed
          saves support).
  \item Run 'make deb'
  \item Finally run 'dpkg -i ../scummvm-cvs*deb', and you're done.
  \end{itemize}
Mac OS X:
\begin{itemize}
\item Make sure you have the developer tools installed.
\item Edit backends/sdl/build.rules, and enable the Mac OS X specific 
      line(s).
\item Depending on where you have installed SDL, you have to add the
      location of its headers to the INCLUDES variables. For example if you
      installed SDL via Fink, you can add this at the end of build.rules:
      INCLUDES+= -I/sw/include
\item You can now 'make' to create a command line binary.
\item To get a version you can run from Finder, type 'make bundle' which
      will create ScummVM.app.
\end{itemize}