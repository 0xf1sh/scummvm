
%%% Local Variables: 
%%% mode: latex
%%% TeX-master: "readme"
%%% End: 

\section{Music and Sound}
By default, on most operating systems, ScummVM will automatically use Adlib
emulation. MIDI may not be available on all operating systems or may need
manual configuration. If you ARE using MIDI, you have several different
choices of output, depending on your operating system and configuration.

\begin{tabular}[h]{ll}
  adlib     & Uses internal Adlib Emulation (default)\\
  mt32      & Uses internal MT-32 Emulation\\
  pcjr      & Uses internal PCjr Emulation \\
  pcspk     & Uses internal PC Speaker Emulation\\
  towns     & Uses FM-TOWNS YM2612 Emulation\\
  windows   & Windows MIDI. Uses built-in sequencer, for Windows users\\
  seq       & Uses /dev/sequencer for MIDI, *nix users. See below.\\
  qt        & Quicktime sound, for Macintosh users.\\
  core      & CoreAudio sound, for MacOS X users.\\
  amidi     & Uses the MorphOS MIDI system, for MorphOS users\\
  alsa      & Output using ALSA sequencer device. See below.\\
  null      & Null output. Don't play any music.\\
\end{tabular}

To select a sound driver, pass its name via the '-e' option to scummvm,
for example:
\begin{verbatim}
   scummvm -e adlib monkey2
\end{verbatim}


%%% Local Variables: 
%%% mode: latex
%%% TeX-master: "readme"
%%% End: 

\subsection{Playing sound with Adlib emulation}

By default an Adlib card will be emulated and ScummVM will output the music
as sampled waves. This is the default mode for most games, and offers the
best compatibility between machines and games.


%%% Local Variables: 
%%% mode: latex
%%% TeX-master: "readme"
%%% End: 

\subsection{Playing sound with MT-32 emulation}

Some games which contain MIDI music data also have improved tracks designed
for MT-32 sound module. ScummVM can now emulate this card, however you should
provide original MT-32 ROMs to make it work. Put the roms in game directory or
directory specified by extrapath.

You don't need to specify --native-mt32 with this driver, as it automatically
gets turned on.

\textbf{NOTE:} You need to have enough processor power to use this emulator as
 it uses heavy floating-point computations.


%%% Local Variables: 
%%% mode: latex
%%% TeX-master: "readme"
%%% End: 

\subsection{Playing sound with MIDI emulation}

Some games (such as Sam and Max) only contain MIDI music data.  This once
prevented music for these games from working on platforms that do not support
MIDI, or soundcards that do not provide MIDI drivers (e.g, many soundcards will
not play MIDI under Linux). ScummVM can now emulate MIDI mode using sampled
waves and Adlib emulation using the -eadlib option.  However, if you are capable
of using native MIDI, we recommend using one of the MIDI modes below for best
sound.


%%% Local Variables: 
%%% mode: latex
%%% TeX-master: "readme"
%%% End: 

\subsection{Playing sound with Native MIDI}
% FIXME: Hardcoding < and > here produces wrong output
Use the appropriate -e<mode> command line option from the list above to
select your preferred MIDI device. For example, if you wish to use the
Windows MIDI driver, use the -ewindows option.


%%% Local Variables: 
%%% mode: latex
%%% TeX-master: "readme"
%%% End: 

\subsection{Playing sound with Sequencer MIDI}

If your soundcard driver supports a sequencer, you may set the environment
variable "SCUMMVM\_MIDI" to your sequencer device - e.g., /dev/sequencer

If you have problems with not hearing audio in this configuration, it is
possible you will need to set the "SCUMMVM\_MIDIPORT" variable to 1 or 2. This
selects the port on the selected sequencer to use. Then start scummvm with the
-eseq parameter. This should work on several cards, and may offer better
performance and quality than Adlib emulation. However, for those systems where
sequencer support does not work, you can always fall back on Adlib emulation.


\subsubsection{Playing sound with ALSA sequencer}

If you have installed the ALSA driver with the sequencer support, then
set the environment variable SCUMMVM\_PORT or the config file parameter
alsa\_port to your sequencer port. The default is "65:0".

Here is a little howto on how to use the ALSA sequencer with your soundcard.
In all cases, to have a list of all the sequencer ports you have, try the
command 
\begin{verbatim}
     aconnect -o -l
\end{verbatim}
This should give output similar to:
\begin{verbatim}
client 64: 'External MIDI 0' [type=kernel]
    0 'MIDI 0-0        '
client 65: 'Emu10k1 WaveTable' [type=kernel]
    0 'Emu10k1 Port 0  '
    1 'Emu10k1 Port 1  '
    2 'Emu10k1 Port 2  '
    3 'Emu10k1 Port 3  '
client 128: 'Client-128' [type=user]
    0 'TiMidity port 0 '
    1 'TiMidity port 1 '
\end{verbatim}
%
This means the external MIDI output of the sound card is located on the
port 64:0, four WaveTable MIDI outputs in 65:0, 65:1, 65:2
and 65:3, and two TiMidity ports, located at 128:0 and 128:1.

If you have a FM-chip on your card, like the SB16, then you have to load
the soundfonts using the sbiload software. \\
Example:
\begin{verbatim}
  sbiload -p 65:0 /etc/std.o3 /etc/drums.o3
\end{verbatim}
%
If you have a WaveTable capable sound card, you have to load a sbk or sf2
soundfont using the sfxload software\\
Example:
\begin{verbatim}
  sfxload /path/to/8mbgmsfx.sf2
\end{verbatim}
%
If you don't have a MIDI capable soundcard, there are two options: FluidSynth
and TiMidity. We recommend FluidSynth, as on many systems TiMidity will 'lag'
behind music. This is very noticable in iMUSE-enabled games, which use fast
and dynamic music transitions. Running TiMidity as root will allow it to
setup real time priority, which may reduce music lag.

Asking TiMidity to become an ALSA sequencer:
\begin{verbatim}
  timidity -iAqqq -B2,8 -Os1S -s 44100 &
\end{verbatim}
If you get distorted output with this setting, you can try dropping the 
-B2,8 or changing the value.

Asking FluidSynth to become an ALSA sequencer (using SoundFonts):
\begin{verbatim}
  fluidsynth -m alsa_seq /path/to/8mbgmsfx.sf2
\end{verbatim}
%
Once either TiMidity or FluidSynth are running, use
\begin{verbatim}
      aconnect -o -l
\end{verbatim}
as described earlier in this section.

%%% Local Variables: 
%%% mode: latex
%%% TeX-master: "readme"
%%% End: 

\subsection{Using compressed audiofiles (MP3, Ogg Vorbis, Flac)}

\subsubsection{Using MP3 files for CD audio}

Use LAME or some other mp3 encoder to rip the cd audio tracks to files. Name
the files track1.mp3 track2.mp3 etc. ScummVM must be compiled with MAD support
to use this option. You'll need to rip the file from the CD as a WAV file,
then encode the MP3 files in constant bit rate. This can be done with the 
following LAME command line:
\begin{verbatim}
  lame -t -q 0 -b 96 track1.wav track1.mp3
\end{verbatim}


\subsubsection{Using Ogg Vorbis files for CD audio}

Use oggenc or some other vorbis encoder to encode the audio tracks to files.
Name the files track1.ogg track2.ogg etc. ScummVM must be compiled with vorbis
support to use this option. You'll need to rip the files from the CD as a WAV
file, then encode the vorbis files. This can be done with the following oggenc
command line with the value after q specifying the desired quality from 0 to 10:
\begin{verbatim}
  oggenc -q 5 track1.wav
\end{verbatim}


\subsubsection{Using Flac files for CD audio}
Use flac or some other flac encoder to encode the audio tracks to files.
Name the files track1.flac track2.flac etc. In your filesystem only allows 
three letter extensions, name the files track1.fla track2.fla etc. 
ScummVM must be compiled with flac support to use this option. You'll need to 
rip the files from the CD as a WAV file, then encode the flac files. This can 
be done with the following flac command line:
\begin{verbatim}
  flac --best track1.wav
\end{verbatim}
%
Remember that the quality is always the same, varying encoder options will only
affect the encoding time and resulting filesize. 


\subsubsection{Compressing MONSTER.SOU with MP3}

You need LAME, and our extract util from the scummvm-tools package to perform
this task, and ScummVM must be compiled with MAD support.
\begin{verbatim}
  extract monster.sou
\end{verbatim}
%
Eventually you will have a much smaller monster.so3 file, copy this file
to your game directory. You can safely remove the monster.sou file.


\subsubsection{Compressing MONSTER.SOU with Ogg Vorbis}

As above, but ScummVM must be compiled with OGG support. Run:
\begin{verbatim}
  extract --vorbis monster.sou
\end{verbatim}
%
This should produce a smaller monster.sog file, which you should copy to your
game directory. Ogg encoding may take a considerable longer amount of time
than MP3, so have a good book handy.


\subsubsection{Compressing MONSTER.SOU with Flac}

As above, but ScummVM must be compiled with Flac support. Run:
\begin{verbatim}
  extract --flac --best -b 1152 monster.sou
\end{verbatim}
%
This should produce a smaller monster.sof file, which you should copy to your
game directory. Remember that the quality is always the same, varying encoder
options will only affect the encoding time and resulting  filesize. Playing
with the blocksize (-b <value>), has the biggest impact on the resulting
filesize -- 1152 seems to be a good value for those kind of soundfiles. Be sure
to read the encoder documentation before you use other values.


\subsubsection{Compressing sfx/speech in Simon the Sorcerer 1 and 2}

Use our simon2mp3 util from the scummvm-tools package to perform this task.
You can choose between multiple target formats, but note that you can only use
each if ScummVM was compiled with the respective decoder support enabled.

\begin{tabular}[h]{ll}
  simon2mp3 effects    &(For simon1acorn)\\
  simon2mp3 simon      &(For simon1acorn)\\
  simon2mp3 effects.voc&(For simon1talkie)\\
  simon2mp3 simon.voc  &(For simon1talkie)\\
  simon2mp3 simon.wav  &(For simon1win)\\
  simon2mp3 simon2.voc &(For simon2talkie)\\
  simon2mp3 simon2.wav &(For simon2win)\\
  simon2mp3 mac        &(For simon2mac)\\
\end{tabular}

For Ogg Vorbis add --vorbis to the options, i.e.
\begin{verbatim}
  simon2mp3 --vorbis
\end{verbatim}
%
For Flac add --flac and optional parameters, i.e.
\begin{verbatim}
  simon2mp3 --flac --best -b 1152 
\end{verbatim}
%
Eventually you will have a much smaller *.mp3, *.ogg or *.fla file, copy this
file to your game dir. You can safely remove the old file.


\subsubsection{Compressing speech/music in Broken Sword 2}

Use our sword2mp3 util rom the scummvm-tools package to perform this task.
You can choose between multiple target formats, but note  that you can only use
each if ScummVM was compiled with the respective decoder support enabled.

\begin{verbatim}
  sword2mp3 speech1.clu
  sword2mp3 music1.clu
\end{verbatim}
%
For Ogg Vorbis add --vorbis to the options, i.e.
\begin{verbatim}
  sword2mp3 --vorbis
\end{verbatim}
%
Eventually you will have a much smaller *.cl3 or *.clg file, copy this file to
your game dir. You can safely remove the old file.

It is possible to use Flac compression by adding the --flac option. However,
the resulting *.clf file will actually be larger than the original.

Please note that sword2mp3 will only work with the four speech/music files in
Broken Sword 2. It will not work with any of the other *.clu files, nor will it
work with the speech files from Broken Sword 1.


%%% Local Variables: 
%%% mode: latex
%%% TeX-master: "readme"
%%% End: 

\subsection{Output sample rate}

The output sample rate tells ScummVM how many sound samples to play per channel
per second. There is much that could be said on this subject, but most of it
would be irrelevant here. The short version is that for most games 22050 Hz is
fine, but in some cases 44100 Hz is preferable. On extremely low-end systems
you may want to use 11025 Hz, but it's unlikely that you have to worry about
that.

To elaborate, most of the sounds ScummVM has to play were sampled at either
22050 Hz or 11025 Hz. Using a higher sample rate will not magically improve the
quality of these sounds. Hence, 22050 Hz is fine.

Some games use CD audio. If you use compressed files for this, they are
probably sampled at 44100 Hz, so for these games that may be a better choice of
sample rate.

When using the Adlib, FM Towns, PC Speaker or IBM PCjr music drivers, ScummVM
is responsible for generating the samples. Usually 22050 Hz will be plenty for
these, but there is at least one piece of Adlib music in Beneath a Steeel Sky
that will sound a lot better at 44100 Hz.

Using frequencies in between is not recommended. For one thing, your sound card
may not support it. In theory, ScummVM should fall back on a sensible frequency
in that case, but don't count on it. More importantly, ScummVM has to resample
all sounds to its output frequency. This is much easier to do well if the
output frequency is a multiple of the original frequency.

