\subsection{Inherit the Earth notes}
In order to run MacOS X wyrmkeep rerelease of the game you will need to copy
over data from the CD to hard drive. If you're on PC then consult

\url{http://www.scummvm.org/documentation.php?view=maccd-howto}

Although it talks about SCUMM games, it describes HFVExplorer utility. Note
that you will need to put speech data "Inherit the Earth Voices" in same
directory as game data which is stored in

\begin{verbatim}
  Inherit the Earth.app/Contents/Resources
\end{verbatim}

For old Mac OS9 release you will need to compile files in MacBinary format,
i.e. they should have both resource and data forks. Just copy all 'ITE *' files.


\subsection{Gobliiins notes}
CD version of Gobliiins contains one big audio track which needs to be ripped
and copied into game directory. See section \ref{sect-compressing-audiofiles}.

\subsection{Maniac Mansion NES notes}
Supported versions are English USA (E), French (F), Swedish (SW) and European
(U). Either use extract\_mm\_nes utility from tools package (but then game
will not be autodetected) or use PRG section of ROM (with first 16 bytes
stripped) with 262144 bytes size. It should be named "Maniac Mansion (XX).prg",
where XX is standard region abbreviation such as E, F, SW or U. If you add
game manually make sure that platform is set to NES.

\subsection{Commodore64 games notes}
Both Maniac Mansion and Zak McKracken run but Maniac Mansion is not yet
playable. Either use extract\_mm\_c64 (but then game will not be autodetected)
or name D64 disks as "maniac1.d64", "maniac2.d64" and "zak1.d64", "zak2.d64"
respectively. If you add game manually make sure that platform is set to
Commodore64.
