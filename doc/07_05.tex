
%%% Local Variables: 
%%% mode: latex
%%% TeX-master: "readme"
%%% End: 

\subsection{Using MP3 files for CD audio}

Use LAME or some other mp3 encoder to rip the cd audio tracks to files. Name
the files track1.mp3 track2.mp3 etc. ScummVM must be compiled with MAD support
to use this option. You'll need to rip the file from the CD as a WAV file,
then encode the MP3 files in constant bit rate. This can be done with the 
following LAME command line:
\begin{verbatim}
  lame -t -q 0 -b 96 track1.wav track1.mp3
\end{verbatim}


\subsubsection{Using Ogg Vorbis files for CD audio}

Use oggenc or some other vorbis encoder to encode the audio tracks to files.
Name the files track1.ogg track2.ogg etc. ScummVM must be compiled with vorbis
support to use this option. You'll need to rip the files from the CD as a WAV
file, then encode the vorbis files. This can be done with the following oggenc
command line with the value after q specifying the desired quality from 0 to 10:
\begin{verbatim}
  oggenc -q 5 track1.wav
\end{verbatim}


\subsubsection{Compressing MONSTER.SOU with MP3}

You need LAME, and our extract util from the scummvm-tools package to perform
this task, and ScummVM must be compiled with MAD support.
\begin{verbatim}
  extract monster.sou
\end{verbatim}
%
Eventually you will have a much smaller monster.so3 file, copy this file
to your game directory. You can safely remove the monster.sou file.


\subsubsection{Compressing MONSTER.SOU with Ogg Vorbis}

As above, but ScummVM must be compiled with OGG support. Run:
\begin{verbatim}
  extract --vorbis monster.sou
\end{verbatim}
%
This should produce a smaller monster.sog file, which you should copy to your
game directory. Ogg encoding may take a considerable longer amount of time
than MP3, so have a good book handy.


\subsubsection{Compressing sfx/speech in Simon the Sorcerer 1 and 2}

Use our simon2mp3 util from the scummvm-tools package to perform
this task, and ScummVM must be compiled with MAD or VORBIS support.\\

\begin{tabular}[h]{ll}
  simon2mp3 effects    &(For simon1acorn)\\
  simon2mp3 simon      &(For simon1acorn)\\
  simon2mp3 effects.voc&(For simon1talkie)\\
  simon2mp3 simon.voc  &(For simon1talkie)\\
  simon2mp3 simon.wav  &(For simon1win)\\
  simon2mp3 simon2.voc &(For simon2talkie)\\
  simon2mp3 simon2.wav &(For simon2win)\\
\end{tabular}

For Ogg Vorbis add --vorbis, i.e.
\begin{verbatim}
  simon2mp3 --vorbis
\end{verbatim}
%
Eventually you will have a much smaller *.mp3 or *.ogg file, copy this
file to your game dir. You can safely remove the old file.
