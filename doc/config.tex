\section{Configuration file}

By default, the configuration file is saved in, and loaded from:
%
\begin{itemize}
\item Windows: \verb#<windir>\scummvm.ini#
\item Unix: \verb#~/.scummvmrc#
\item Mac OS X: \verb#~/Library/Preferences/ScummVM Preferences#
\item Others: \verb#scummvm.ini# in the current directory
\end{itemize}
%
An example config file looks as follows:
%
\begin{verbatim}
        [scummvm]
        gfx_mode=supereagle
        fullscreen=true
        savepath=C:\saves\

        [sky]
        path=C:\games\SteelSky\

        [germansky]
        gameid=sky
        language=de
        path=C:\games\SteelSky\
        description=Beneath a Steel Sky w/ German subtitles
        
        [germandott]
        gameid=tentacle
        path=C:\german\tentacle\
        description=German version of DOTT

        [tentacle]
        path=C:\tentacle\
        subtitles=true
        music_volume=40
        sfx_volume=255

        [loomcd]
        cdrom=1
        path=C:\loom\
        talkspeed=5
        savepath=C:\loom\saves\
        
        [monkey2]
        path=C:\amiga_mi2\
        music_driver=windows
\end{verbatim}
%
The following keywords are recognized:

\begin{tabular}[hf]{lll}
        basename       &string\\
        path           &string   The path to where a game's data files are\\
        read\_only     &bool     If true, ScummVM will never try to overwrite\\
                       &         the configuration file.\\
        autosave\_period &number   The seconds between autosaving (default: 300)\\
        save\_slot     &number   The save game number to load on startup.\\
        savepath       &string   The path to where a game will store its\\
                       &         savegames.\\
        versioninfo    &string   The version of the ScummVM that created the\\
                       &         configuration file.\\
\\
        gameid         &string   The real id of a game. Useful if you have\\
                       &         several versions of the same game, and want\\
                       &         different aliases for them. See the example.\\
        description    &string   The description of the game as it will appear\\
                       &         in the launcher.\\
\\
        language       &string   Specify language (en, us, de, fr, it, pt, es, jp,\\
                       &         zh, kr, se, gb, hb, cz, ru)\\
        speech\_mute    &bool    If true, speech is muted\\
        subtitles      &bool     Set to true to enable subtitles.\\
        talkspeed      &number   Text speed\\
\\
        fullscreen     &bool     Fullscreen mode\\
        aspect\_ratio  &bool     Enable aspect ratio correction\\
        gfx\_mode      &string   Graphics mode (normal, 2x, 3x, 2xsai,\\
                       &         super2xsai, supereagle, advmame2x, advmame3x,\\
                       &         hq2x, hq3x, tv2x, dotmatrix)\\
\\
        cdrom           &number   Number of CD-ROM unit to use for audio. If\\
                        &         negative, don't even try to access the CD-ROM.\\
        joystick\_num   &number   Number of joystick device to use for input\\
        music\_driver   &string   The music engine to use.\\
        output\_rate    &number   The output sample rate to use, in Hz. Sensible\\
                        &         values are 11025, 22050 and 44100.\\
        alsa\_port      &string   Port to use for output when using the\\
                        &         ALSA music driver.\\
        music\_volume   &number   The music volume setting (0-255)\\
        multi\_midi     &bool     If true, enable combination Adlib and native\\
                        &         MIDI.\\
        soundfont       &string   The SoundFont to use for MIDI playback. (Only\\
                        &         supported by some MIDI drivers.)\\
        native\_mt32    &bool     If true, disable GM emulation and assume that\\
                        &         there is a true Roland MT-32 available.\\
        enable\_gs      &bool     If true, enable Roland GS-specific features to\\
                        &         enhance GM emulation. If native\_mt32 is also\\
                        &         true, the GS device will select an MT-32 map\\
                        &         to play the correct instruments.\\
        sfx\_volume     &number   The sfx volume setting (0-255)\\
        tempo           &number   The music tempo (50-200) (default: 100)\\
        speech\_volume  &number   The speech volume setting (0-255)\\
        midi\_gain      &number   The MIDI gain (0-1000) (default: 100) (Only\\
                        &         supported by some MIDI drivers.)\\
\\
        copy\_protection&bool     Enable copy protection in SCUMM games, when\\
                        &         ScummVM disables it by default.\\
        demo\_mode      &bool     Start demo in Maniac Mansion\\
        alt\_intro      &bool     Use alternative intro for CD versions of \\
                        &         Beneath a Steel Sky and Flight of the Amazon\\
                        &         Queen
\\
        boot\_param     &number   Pass this number to the boot script\\
\end{tabular}

Broken Sword 2 adds the following non-standard keywords:\\
\begin{tabular}[h]{lll}
        gfx\_details    &number  &Graphics details setting (0-3)\\
        music\_mute     &bool    &If true, music is muted\\
        object\_labels  &bool    &If true, object labels are enabled\\
        reverse\_stereo &bool    &If true, stereo channels are reversed\\
        sfx\_mute       &bool    &If true, sound effects are muted\\
\end{tabular}

Flight of the Amazon Queen adds the following non-standard keywords:\\
\begin{tabular}[h]{lll}
        music\_mute     &bool    &If true, music is muted\\
        sfx\_mute       &bool    &If true, sound effects are muted\\
\end{tabular}

Simon the Sorcerer 1 and 2 add the following non-standard keywords:\\
\begin{tabular}[h]{lll}
        music\_mute     &bool    &If true, music is muted\\
        sfx\_mute       &bool    &If true, sound effects are muted\\
\end{tabular}

The Legend of Kyrandia adds the following non-standard keyword:\\
\begin{tabular}[h]{lll}
        walkspeed       &int     &The walk speed (0-4)\\
\end{tabular}


%%% Local Variables: 
%%% mode: latex
%%% TeX-master: "readme.tex"
%%% End: 
