
%%% Local Variables: 
%%% mode: latex
%%% TeX-master: "readme.tex"
%%% End: 


\subsection{Multi-CD games notes}
In general, ScummVM does not deal very well with Multi-CD games. This is
because ScummVM assumes everything about a game can be found in one directory.
Even if ScummVM does make some provisions for asking the user to change CD, the
original games usually install a small number of files to the hard disk. Unless
these files can be found on all the CDs, ScummVM will be in trouble.

Fortunately, ScummVM has no problems running the games entirely from hard disk,
if you create a directory with the correct combination of files. Usually, when
a file appears on more than one CD you can pick either of them.

These instructions are written for the PC versions (which in some case is the
only version) of the games. Windows and DOS use case-insensitive file systems,
so if one CD has a file called MONKEY.DAT and another has a file called
monkey.dat, they are the same files. These instructions give file names in all
lower-case names, even if that's not always how they appear on the CDs. In
fact, on case-sensitive file systems you will have to make sure that all
filenames use either all upper- or all lower-case letters for ScummVM to be
able to find the files.

\subsubsection{The Curse of Monkey Island notes}
For this game, you'll need the comi.la0, comi.la1 and comi.la2 files. The
comi.la0 file can be found on either CD, but since they are identical it
doesn't matter which one of them you use.

In addition, you'll need a resource subdirectory with all of the files from the
resource subdirectories on both CDs. Some of the files appear on both CDs, but
again they're identical.
