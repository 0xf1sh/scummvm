
%%% Local Variables: 
%%% mode: latex
%%% TeX-master: "readme"
%%% End: 

\subsection{Playing sound with FluidSynth MIDI emulation}

If ScummVM was build with libfluildsynth support it will be able to play MIDI
music through the FluidSynth driver. You will have to specify a SoundFont to
use, however.

Since the default output volume from FluidSynth can be fairly low, ScummVM will
set the gain by default to get a stronger signal. This can be further adjusted
using the --midi-gain command-line option, or the ``midi\_gain'' config file
setting.

The setting can take any value from 0 through 1000, with the default being 100.
(This corresponds to FluidSynth's gain settings of 0.0 through 10.0, which are
presumably measured in decibel.)

\textbf{NOTE:} The processor requirements for FluidSynth can be fairly high in
some cases. A fast CPU is recommended.
