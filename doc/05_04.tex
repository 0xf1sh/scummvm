
%%% Local Variables: 
%%% mode: latex
%%% TeX-master: "readme"
%%% End: 

\subsection{Hot Keys}
ScummVM supports various in game hotkeys. They differ between the SCUMM and
Simon games.
\begin{itemize}
\item Common:\\
  \begin{tabular}{ll}
    Ctrl-z OR Alt-x        & Quit\\
    Keyboard Arrow Keys    & Simulate mouse movement\\
    Ctrl-f                 & Toggle fast mode\\
    Ctrl-m                 & Toggle mouse capture\\
    Ctrl-Alt 1-8           & Switch between graphics filters\\
    Ctrl-Alt + and -       & Increase/Decrease the scale factor\\
    Ctrl-Alt a             & Toggle aspect-ratio correction on/off\\
                           & Most of the games use a 320x200 pixel\\
                           & resolution, which may look squashed on\\
                           & modern monitors. Aspect-ratio correction\\
                           & stretches the image to use 320x240 pixels\\
                           & instead, or a multiple thereof\\
    Alt-Enter              & Toggles full screen/windowed\\
  \end{tabular}
\item Scumm:\\
  \begin{tabular}{ll}
    Ctrl 0-9 and Alt 0-9   & Load and save game state\\
    Ctrl-g                 & Runs in really REALLY fast mode\\
    Ctrl-d                 & Starts the debugger\\
    Tilde \verb#~#         & Show/hide the debugging console\\
    Ctrl-s                 & Shows memory consumption\\
    $[$ and $]$                & Music volume, down/up\\
    - and +                & Text speed, slower/faster\\
    F5                     & Displays a save/load box\\
    Space                  & Pauses\\
    Period (.)             & Skips current line of text in some games\\
    Enter                  & Simulate left mouse button press\\
    Tab                    & Simulate right mouse button press\\
  \end{tabular}
\item Beneath a Steel Sky:\\
  \begin{tabular}{ll}
    Ctrl 0-9 and Alt 0-9   & Load and save game state\\
    Ctrl-g                 & Runs in really REALLY fast mode\\
    F5                     & Displays a save/load box\\
    Escape                 & Skips the game intro\\
    Period (.)             & Skips current line of text\\
  \end{tabular}
\item Broken Sword I:\\
  \begin{tabular}{ll}
    F5                     & Displays save/load box\\
  \end{tabular}
\item Broken Sword II:\\
  \begin{tabular}{ll}
    Ctrl-d                 & Starts the debugger\\
    c                      & Displays the credits\\
    p                      & Pauses\\
  \end{tabular}
\item Flight of the Amazon Queen:
  \begin{tabular}{ll}
    Ctrl-d                 & Starts the debugger\\
  \end{tabular}
\item Simon the Sorcerer 1 \& 2:\\
  \begin{tabular}{ll}
    Ctrl 0-9 and Alt 0-9   & Load and save game state\\
    Ctrl-d                 & Starts the debugger\\
    F1 - F3                & Text speed, faster - slower\\
    F10                    & Shows all characters and objects you can \\
                           & interact with\\
    - and +                & Music volume, down/up\\
    m                      & Music on/off\\
    s                      & Sound effects on/off\\
    b                      & Background sounds on/off\\
    p                      & Toggles pause\\
    t                      & Switch between speech and subtitles\\
    v                      & Switch between subtitles only and\\
                           & combined speech \& subtitles\\
                           & (Simon the Sorcerer 2 only)\\
  \end{tabular}
\end{itemize}
Note that using ctrl-f and ctrl-g are not recommended: games can crash when
being ran faster than their normal speed, as scripts will lose synchronisation.

Ctrl-f is not supported by the Broken Sword games.
%%% Local Variables: 
%%% mode: latex
%%% TeX-master: "readme"
%%% End: 
