\subsection{Using compressed audiofiles (MP3, Ogg Vorbis, Flac)}
\label{sect-compressing-audiofiles}


\subsubsection{Using MP3 files for CD audio}

Use LAME or some other MP3 encoder to rip the CD audio tracks to files. Name
the files track1.mp3 track2.mp3 etc. ScummVM must be compiled with MAD support
to use this option. You'll need to rip the file from the CD as a WAV file,
then encode the MP3 files in constant bit rate. This can be done with the
following LAME command line:
\begin{verbatim}
  lame -t -q 0 -b 96 track1.wav track1.mp3
\end{verbatim}


\subsubsection{Using Ogg Vorbis files for CD audio}

Use oggenc or some other vorbis encoder to encode the audio tracks to files.
Name the files track1.ogg track2.ogg etc. ScummVM must be compiled with vorbis
support to use this option. You'll need to rip the files from the CD as a WAV
file, then encode the vorbis files. This can be done with the following oggenc
command line with the value after q specifying the desired quality from 0 to 10:
\begin{verbatim}
  oggenc -q 5 track1.wav
\end{verbatim}


\subsubsection{Using Flac files for CD audio}

Use flac or some other flac encoder to encode the audio tracks to files.
Name the files track1.flac track2.flac etc. In your filesystem only allows
three letter extensions, name the files track1.fla track2.fla etc.
ScummVM must be compiled with flac support to use this option. You'll need to
rip the files from the CD as a WAV file, then encode the flac files. This can
be done with the following flac command line:
\begin{verbatim}
  flac --best track1.wav
\end{verbatim}
%
Remember that the quality is always the same, varying encoder options will only
affect the encoding time and resulting filesize.


\subsubsection{Compressing MONSTER.SOU with MP3}

You need LAME, and our compress\_scumm\_sou utility from the scummvm-tools
package to perform this task, and ScummVM must be compiled with MAD support.
\begin{verbatim}
  compress_scumm_sou monster.sou
\end{verbatim}
%
Eventually you will have a much smaller monster.so3 file, copy this file
to your game directory. You can safely remove the monster.sou file.


\subsubsection{Compressing MONSTER.SOU with Ogg Vorbis}

As above, but ScummVM must be compiled with OGG support. Run:
\begin{verbatim}
  compress_scumm_sou --vorbis monster.sou
\end{verbatim}
%
This should produce a smaller monster.sog file, which you should copy to your
game directory. Ogg encoding may take a considerable longer amount of time
than MP3, so have a good book handy.


\subsubsection{Compressing MONSTER.SOU with Flac}

As above, but ScummVM must be compiled with Flac support. Run:
\begin{verbatim}
  compress_scumm_sou --flac --best -b 1152 monster.sou
\end{verbatim}
%
This should produce a smaller monster.sof file, which you should copy to your
game directory. Remember that the quality is always the same, varying encoder
options will only affect the encoding time and resulting  filesize. Playing
with the blocksize (-b <value>), has the biggest impact on the resulting
filesize -- 1152 seems to be a good value for those kind of soundfiles. Be sure
to read the encoder documentation before you use other values.


\subsubsection{Compressing sfx/speech in Simon the Sorcerer 1 and 2}

Use our compress\_simon util from the scummvm-tools package to perform this task.
You can choose between multiple target formats, but note that you can only use
each if ScummVM was compiled with the respective decoder support enabled.

\begin{tabular}[h]{ll}
  compress\_simon effects    &(For Acorn CD version of Simon 1)\\
  compress\_simon simon      &(For Acorn CD version of Simon 1)\\
  compress\_simon effects.voc&(For DOS CD version of Simon 1)\\
  compress\_simon simon.voc  &(For DOS CD version of Simon 1)\\
  compress\_simon simon.wav  &(For Windows CD version of Simon 1)\\
  compress\_simon simon2.voc &(For DOS CD version of Simon 2)\\
  compress\_simon simon2.wav &(For Windows CD version of Simon 2)\\
  compress\_simon mac        &(For Macintosh version of Simon 2)\\
\end{tabular}

For Ogg Vorbis add --vorbis to the options, i.e.
\begin{verbatim}
  compress_simon --vorbis
\end{verbatim}
%
For Flac add --flac and optional parameters, i.e.
\begin{verbatim}
  compress_simon --flac --best -b 1152
\end{verbatim}
%
Eventually you will have a much smaller *.mp3, *.ogg or *.fla file, copy this
file to your game directory. You can safely remove the old file.


\subsubsection{Compressing speech/music in Broken Sword 1}

The compress\_sword1 tool from the scummvm-tools package can encode music and
speech to MP3 as well as Ogg Vorbis.
The easiest way to encode the files is simply copying the executable into your
BS1 directory (together with the lame encoder) and run it from there.
This way, it'll automatically encode everything to MP3.
Afterwards, you can manually remove the SPEECH?.CLU files and the wave music
files.

Running
\begin{verbatim}
  compress_sword1 --vorbis
\end{verbatim}
%
will compress the files using Ogg Vorbis instead of MP3.

Use
\begin{verbatim}
  compress_sword1 --help
\end{verbatim}
%
to get a full list of the options.


\subsubsection{Compressing speech/music in Broken Sword 2}

Use our compress\_sword2 util from the scummvm-tools package to perform this
task. You can choose between multiple target formats, but note  that you can
only use each if ScummVM was compiled with the respective decoder support
enabled.

\begin{verbatim}
  compress_sword2 speech1.clu
  compress_sword2 music1.clu
\end{verbatim}
%
For Ogg Vorbis add --vorbis to the options, i.e.
\begin{verbatim}
  compress_sword2 --vorbis
\end{verbatim}
%
Eventually you will have a much smaller *.cl3 or *.clg file, copy this file to
your game directory. You can safely remove the old file.

It is possible to use Flac compression by adding the --flac option. However,
the resulting *.clf file will actually be larger than the original.

Please note that compress\_sword2 will only work with the four speech/music
files in Broken Sword 2. It will not work with any of the other *.clu files,
nor will it work with the speech files from Broken Sword 1.


%%% Local Variables: 
%%% mode: latex
%%% TeX-master: "readme.tex"
%%% End: 
