\subsection{Graphics filters} \label{sect-gfx-filters}

ScummVM offers several anti-aliasing filters to attempt to improve visual
quality. These are the same filters used in many other emulators, such as
MAME. These filters take the original game graphics, and scale it by a
certain fixed factor (usually 2x or 3x) before displaying them to you.
So for example, if the game originally run at a resolution of 320x200
(typical for most of the SCUMM games), then using a filter with scale
factor 2x will effectively yield 640x400 graphics. Likewise with a
3x filter you'll get 960x600.

They are:\\
\begin{tabular}[h]{ll}
  1x         & No filtering, no scaling. Fastest.\\
  2x         & No filtering, factor 2x (default for non 640x480 games).\\
  3x         & No filtering, factor 3x.\\
  2xsai      & 2xSAI filter, factor 2x.\\
  super2xsai & Enhanced 2xSAI filtering, factor 2x.\\
  supereagle & Less blurry than 2xSAI, but slower. Factor 2x.\\
  advmame2x  & Doesn't rely on blurring like 2xSAI, fast. Factor 2x.\\
  advmame3x  & Doesn't rely on blurring like 2xSAI, fast. Factor 3x.\\
  hq2x       & Very nice high quality filter but slow. Factor 2x.\\
  hq3x       & Very nice high quality filter but slow. Factor 3x.\\
  tv2x       & Interlace filter, tries to emulate a TV. Factor 2x.\\
  dotmatrix  & Dot matrix effect. Factor 2x.\\
\end{tabular}

To select a graphics filter, pass its name via the '-g' option to scummvm,
for example:

\begin{verbatim}
  scummvm -gadvmame2x monkey2
\end{verbatim}
\begin{enumerate}
\item [Note \#1] Not all backends support all or any filters. The ones
  listed above are for the default SDL backend.
\item [Note \#2] Filters can be very slow when ScummVM is compiled in a
  debug configuration without optimizations. And there is always a
  speed impact when using any form of anti-aliasing/linear filtering.
\item [Note \#3] The FM-TOWNS version of Zak McKracken uses an
  original resolution of 320x240, hence for this game scalers will
  scale to 640x480 or 960x720.
\end{enumerate}


%%% Local Variables: 
%%% mode: latex
%%% TeX-master: "readme.tex"
%%% End: 
